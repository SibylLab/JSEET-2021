\section{Board and Card Games That Teach Programming}

This section provides a survey of other efforts for teaching computer programming where the medium used is a card game or a board game. 
Based on these works, we found that practical implementation (i.e. learning a real programming language concept) and visualization of a program's results are essential elements in learning computer programming. Also, we found that although learning through gameplay may have more sustainable effects \cite{effectivenessofgames}, the main challenge is that of maintaining a balance between learning and engagement \cite{educationalgamedesign}. Finally, the gameplay needs to be presented in a structured way such that the user can relate the outcome to real-world programming.

\paragraph{Potato Pirates:}
Potato Pirates \cite{PotatoPirates} is perhaps the most similar game to \gameName in intention and execution. It is a physical card game that intends to teach all the essential programming concepts. The premise of the game is that the player is a potato who captains a pirate ship, and the player conducts naval battles to sink the other player's ships and assimilate their crew. The goal of the game is to acquire seven ``Potato King" cards by performing different actions with playable cards. By playing the game, the user becomes familiar with different programming concepts. Players can power up their attacks with cards that represent loops and conditionals. This includes cards to represent different structures such as for-loops and while-loops, as well as an option to create nested-loops. Some cards introduce the concepts of interrupts and control flow. Finally, the game covers the concepts of variables, functions, if-else conditionals, and the switch-case construct.

\paragraph{Battle Bots v2:}
Battle Bots v2 \cite{BattleBotsv2}  was inspired by Robo Rally \cite{roborally}. In the game, each move is divided into two groups: the programming round and the action round. In the programming round, players play movement cards to move their ``bot" around the board. In the action round, there are five phases where different action and movement cards can be played. A player who survives the attacks from their opponents wins the game. Players need to design their gameplay according to the other players’ moves or personal requirements. The game teaches the logical aspects of computer programming through the players need to plan their moves. One of the game's drawbacks is that it has no actual programming interface. Also, it can be hard to understand how the game is to be played as we found that the description of the rules is not very clear.


\paragraph{Robot Turtles:}
Robot Turtles \cite{RobotTurtles} is a game that teaches computer programming by using simple direction cards to move a coloured turtle and is intended for pre-school age children. The game's objective is to move a coloured turtle from a square on the edge of the board to a square with a jewel matching the turtle's colour. The game's use of a single card for either directional or rotational movement of the turtle works well to simulate single statement programming instructions. This is effective at providing children with their first look at programming. Additionally, Robot Turtles introduces the concept of function calls with the \texttt{Jump} card which can be used to replace a set of instruction cards. These replaced cards can then be used repeatedly. Finally, the game introduces players to debugging through a \texttt{Bug} card that can be added to a player's program to indicate that it has a problem.

\paragraph{Code Master:}
Code Master \cite{CodeMaster} is a single-player puzzle game that teaches logical problem-solving. The goal of the game is to traverse a graph from a starting node to a destination node while using different moves to collect crystals along the way. The game introduces a player to the programming constructs of conditional statements (i.e. if-else), repetition (some nodes have a loop) and finding the shortest path. One of the game's primary focuses is critical thinking, which is a necessary skill for computer programming.

\paragraph{Blocky Maze:}
Blocky Maze \cite{Blocklymaze} is designed to teach the concepts of loops and conditions to players with no prior computer programming experience. The game uses a visual programming language that can be compiled into either JavaScript, Dart or Python to execute the program. In the game, programming is done by dragging and dropping code blocks onto a design surface. The main objective of the game is to take the player's avatar from a start point to an endpoint. The game is made up of a total of ten levels which increase in complexity as the player progresses.

